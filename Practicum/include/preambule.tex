% -*-coding: utf-8 -*-
\documentclass[11pt, a4paper, spanish]{article}
\usepackage[spanish]{babel}
\usepackage[utf8]{inputenc}
\usepackage{xcolor}
\usepackage[
  colorlinks = true,
  linkcolor = black,
  urlcolor  = blue,
  citecolor = blue,
  anchorcolor = blue,
  pdfauthor={\Author},
  pdftitle={\DocumentTitle},
  pdfborder={0 0 0}]{hyperref}
\usepackage[T1]{fontenc}
\usepackage{longtable}
\usepackage{url}
\usepackage[stable]{footmisc}
\usepackage{parskip}
\usepackage{setspace}
\usepackage{amsfonts,amsgen,amsmath,amssymb}

% Used to include section of different files avoiding the \clearpage of \include
\usepackage{newclude}

% Multicol package
\usepackage{multicol}

% Custom depth of section on Table of Contents
\setcounter{tocdepth}{4}
\setcounter{secnumdepth}{4}

% Images packages
\usepackage[pdftex]{graphicx}
\usepackage{float}
\usepackage{subfig}
\usepackage{color}
\DeclareGraphicsExtensions{.png,.jpg,.pdf,.mps,.gif,.bmp}

% Margin of pages
\usepackage{geometry}
\geometry{
 a4paper,
 right=20mm,
 bottom=30mm,
 left=20mm,
 top=20mm,
}

% Header config.
\usepackage{fancyhdr}
\fancyhead[L]{\today}
\fancyhead[R]{\ResearchGroup}
\fancyfoot[C]{\rule{1cm}{0.5mm}\\\thepage} \pagestyle{fancy}
\setlength{\headheight}{20pt}

% Establish indentation (idk why doesn't work by default)
\setlength\parindent{6mm}

% Custom line for title
\newcommand{\HRule}{\rule{\linewidth}{0.5mm}}
