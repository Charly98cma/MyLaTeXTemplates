\documentclass[12pt,a4paper, spanish]{article}
\usepackage[spanish]{babel}
\usepackage{setspace}
\usepackage{float} % Usado par aponer imagenes una al lado de la otra
\usepackage[ % Informacion del usuario en los metadatos del pdf creado (CAMBIAR POR EL USUARIO)
  pdftex,
  pdfauthor={AUTOR DEL DOCUMENTO},
  pdftitle={TITULO DEL DOCUMENTO},
  pdfsubject={ASUNTO DEL DOCUMENTO}]{hyperref}
% --- PAQUETES PARA USAR IMAGENES ---
\usepackage[pdftex]{graphicx}
\usepackage{subfig}
\usepackage{graphicx}
\usepackage[usenames,dvipsnames]{color}
\DeclareGraphicsExtensions{.png,.jpg,.pdf,.mps,.gif,.bmp}
% --- END IMAGENES ---

% --- DIMENSIONES DE LOS MARGENES ---
\frenchspacing \addtolength{\hoffset}{-1.5cm}
\addtolength{\textwidth}{3cm} \addtolength{\voffset}{-2.5cm}
\addtolength{\textheight}{4cm}
% --- END DIMENSIONES MARGENES ---

% --- ENCABEZADO ---
\usepackage{fancyhdr}
\fancyhead[R]{**INTRODUCIR ASIGNATURA**}\fancyhead[L]{\today} % Rellenar con la asignatura del trabajo a realizar
\fancyfoot[C]{\thepage} \pagestyle{fancy}
% --- END ENCABEZADO ---


\begin{document}


% --- PORTADA ---
\begin{titlepage}
  \newcommand{\HRule}{\rule{\linewidth}{0.5mm}}
  \centering

  % --- TITULOS SECUNDARIOS ---
  \textsc{}\\[0.25cm]

  \textsc{\huge Universidad Politécnica de Madrid}\\[0.5cm]

  \textsc{\LARGE Escuela Técnica Superior\\ de Ingeniería Informática}\\[0.3cm]

  \begin{figure}[H] % H -> emplaza las imagenes en el mismo lugar en el que se encuentran en el codigo
    \centering
    \subfloat{{\includegraphics[width=6cm]{LogoUPM2.png}}}
    \qquad
    \subfloat{{\includegraphics[width=2.75cm]{LogoFi.png}}}\\[0.5cm]
  \end{figure}

  \textsc{\Large **INTRODUCIR DEPARTAMENTO**}\\[0.25cm]
  \textsc{\large **INTRODUCIR ASIGNATURA**}\\[0.25cm]

  % --- TITULO PRINCIPAL ---
  \HRule\\[0.4cm]

  {\huge\bfseries INTRODUCIR TÍTULO\\[0.4cm] INTRODUCIR TITULO}\\[0.4cm]

  \HRule\\[1.25cm]

  % --- DATOS CURSO Y SEMESTRE ---

  \textsc{\large XX CURSO XXX SEMESTRE}\\[1.5cm]

  % --- AUTORES ---
  {\large\underline{\textit{Autor/es:}}}\\[0.2cm]
     \textsc{APELLIDOS, NOMBRE}\\
             Nº de matrícula: XXXXXX\\
     [0.5cm]
     \textsc{APELLIDOS, NOMBRE}\\
             Nº de matrícula: XXXXXX\\[0.5cm]

   % --- FECHA ---
   \vfill\vfill\vfill
   {\large\today}

\end{titlepage}
% --- END PORTADA ---


\newpage
% --- INDICE ---
\pagenumbering{gobble} % No se numera la pagina del indice
\renewcommand*\contentsname{Índice de contenidos} % Nombre del indice (a elegir por el creador)
\tableofcontents
% --- END Indice ---


\newpage
\pagenumbering{arabic} % Nmeracion de páginas a partir de este punto.

\section{SECCIÓN 1}

\subsection{SUBSECCIÓN 1}



\end{document}
